\documentclass[12pt,one side]{article}\usepackage[utf8]{inputenc}\usepackage[a4 paper]{geometry}\usepackage{graphicx, setspace, appendix, mathrsfs, enumerate,amsmath, amsfonts, array, tabularx, longtable, rotating, caption, mathtools}\usepackage[english]{babel}\renewcommand{\baselinestretch}{1.25}\doublespacing\title{}\author{Yao Luo}\begin{document}\maketitle

\textbf{\textit{Assumption 1}}: Two firms in the market, compete in price, and the product is identical.\\
Potential extension: generalize to N firms.\\

\textbf{\textit{Assumption 2}}: Firms have the same constant marginal cost and is normalized to zero.\\
Potential extension: \\
(1)Firms independently draw from a known distribution of marginal cost.\\
(2)Firms are randomly selected to have high($c_H$) or low($c_L$) marginal costs.\\

\textbf{\textit{Assumption 3}}: Firms' profit function is \[\pi(p,q) = pq(p)\]
where $q(p)$ is consumers' demand function and marginal cost is zero. For now assume $q(p) = a-bp$\\

\textbf{\textit{Assumption 4}}: At the first stage, firms simultaneously decide whether to adopt price-matching guarantees(PMG) policy or not. The decision will then be observed by both firms and all consumers. At the second stage, firms choose their list prices simultaneously. \\
Potential extension: After allowing for random marginal cost, consumers may or may not observe it but firms observe it.\\

\textbf{\textit{Assumption 5}}: Consumers' utility function is: \[u_{ij} = \alpha_{ij} - \beta p_j -\gamma s_{i}+ \epsilon_{ij}\]
where $\alpha_{ij}$ measures consumer $i$'s specific taste for firm $j$. $p_j$ is firm $j$'s list price. $s_{i}$ is consumer's search cost. And $\epsilon_{ij}$ is observed to the consumers but unobserved to the researcher. For now assume linear utility function. Will change it according to the consumer segmentation I assumed.\\

\textbf{\textit{Assumption 6}}: Consumer segmentation. Potential options I consider so far are:\\
(1) Continuous transportation cost.\\
(2) Searching + switching costs. Some consumers are loyal so will always buy the product from one firm. Some are not loyal. Searching costs are different, range from zero to infinity.\\
(3) Consumers may have different search costs for both pre-purchase and post-purchase.\\

\textbf{\textit{Assumption 7}}: Some regulation conditions to make sure all consumers consume in equilibrium.\\

\textbf{\textit{Assumption 8}}: Regulation conditions on demand functions facing the firms.\\
\end{document}