\documentclass[dvipsnames,mathserif]{beamer}
\setbeamertemplate{footline}[frame number]
\setbeamercolor{footline}{fg=black}
\setbeamerfont{footline}{series=\bfseries}
\usepackage{tikz}
\usepackage{xcolor}
\usepackage{graphicx, setspace, appendix, mathrsfs, amsmath, amsfonts,caption, mathtools}
\usepackage[english]{babel}
%\usetheme{Frankfurt}%1
\usetheme{Darmstadt}%1

% for RTL liste
\makeatletter
\newcommand{\RTListe}{\raggedleft\rightskip\leftm}
\newcommand{\leftm}{\@totalleftmargin}
\makeatother

% RTL frame title
\setbeamertemplate{frametitle}
{\vspace*{-1mm}
  \nointerlineskip
    \begin{beamercolorbox}[sep=0.3cm,ht=2.2em,wd=\paperwidth]{frametitle}
        \vbox{}\vskip-2ex%
        \strut\hskip1ex\insertframetitle\strut
        \vskip-0.8ex%
    \end{beamercolorbox}
}
% align subsection in toc
\makeatletter
\setbeamertemplate{subsection in toc}
{\leavevmode\rightskip=5ex%
  \llap{\raise0.1ex\beamer@usesphere{subsection number projected}{bigsphere}\kern1ex}%
  \inserttocsubsection\par%
}
\makeatother

% RTL triangle for itemize
\setbeamertemplate{itemize item}{\scriptsize\raise1.25pt\hbox{\donotcoloroutermaths$\blacktriangleleft$}} 

%\setbeamertemplate{itemize item}{\rule{4pt}{4pt}}

\defbeamertemplate{enumerate item}{square2}
{\LR{
    %
    \hbox{%
    \usebeamerfont*{item projected}%
    \usebeamercolor[bg]{item projected}%
    \vrule width2.25ex height1.85ex depth.4ex%
    \hskip-2.25ex%
    \hbox to2.25ex{%
      \hfil%
      {\color{fg}\insertenumlabel}%
      \hfil}%
  }%
}}

\setbeamertemplate{enumerate item}[square2]
\setbeamertemplate{navigation symbols}{}


\titlegraphic { 
\begin{tikzpicture}[overlay,remember picture, opacity=0.1,]
\node[] at (0, 2.9){
};\end{tikzpicture}}
\setbeamertemplate{caption}[numbered]
\begin{document}

\rightskip\rightmargin
\title{POLICE OFFICER ASSIGNMENT AND NEIGHBORHOOD CRIME}
\author{Ba et al, presented by Yao Luo}


\institute{Boston College}
\footnotesize{\date{\today }


\begin{frame}
\maketitle
\end{frame}


%
\footnotesize \tableofcontents
%
\section{Introduction}
\begin{frame}{Introduction}
    \begin{itemize}
    	\item This paper studies the impact of chicago's seniority-based mechanism for allocating police officers to districts.\\
    	\vspace{0.2cm}
    	\item Experienced officers self select into safer districts. They also deter violent crime and use less force.\\
    	\vspace{0.2cm}
    	\item Key contribution: the first to examine the economic implications of police officer assignment mechanisms empirically.\\
    	\vspace{0.2cm}
    	\item Model: officer preferences + crime production function.\\
    	\vspace{0.2cm}
    	\item Counterfactual: offer subsidies to incentivize offciers to choose less desirable districts.
        
    \end{itemize}
\end{frame}

\section{Findings}
\begin{frame}{Findings}
    \begin{itemize}
    	\item Model: increasing the share of more experienced officers in a district reduces violent crime, while having a negligible impact on property crime.\\
    	\vspace{0.5cm}
    	\item Counterfactual: equalizing tenure across districs leads to a sizable decline in the aggregate violent crime rate while having little impact on property crime. There is a substantial reduction of violent crime in the highest crime districts.\\
        
    \end{itemize}
\end{frame}

\section{Discussion}
\begin{frame}{Discussion}
    \begin{itemize}
    	\item This paper sovles the endogeneity issue very cleverly: exploiting the rigid structure of police 4+2 shift schedule and using the simulated IV.\\
    	\vspace{0.5cm}
    	\item Is it sensible to assume that the police officers are risk-neutral? Does is make sense to assume linear additively seperable utility function? Partial identification good enough?
        
    \end{itemize}
\end{frame}
\end{document}