\documentclass[12pt, one side]{article}\usepackage[utf8]{inputenc}\usepackage[a4 paper]{geometry}\usepackage{graphicx, setspace, appendix, mathrsfs, enumerate,amsmath, amsfonts, array, tabularx, longtable, rotating, caption, mathtools}\usepackage[english]{babel}\renewcommand{\baselinestretch}{1.25}\title{Reverse Outline of the Paper "Equilibrium uniqueness in entry games with private information"}\author{Yao Luo}\begin{document}\maketitle
\section{Introduction}
\begin{enumerate}
	\item Understanding firms’ market entry decisions is a key element of economic policy and regulation.
	\item We study equilibrium uniqueness in static binary-action entry games with single- dimensional private information.
	\item We characterize firms’ equilibrium behavior using a simple index, called strength, sum- marizing a firm’s ability to endure competition.
	\item Our proposed framework encompasses static entry models commonly used in applied work.
	\item Some theoretical literature on market entry.
	\item Some empirical literature on market entry.
	\item Why it is important to allow for private information in entry models.
	\item The structure of the article.
\end{enumerate}
\section{An illustrative example}
\begin{enumerate}
	\item Second-price auction with entry costs.
	\begin{enumerate}
		\item Set up.
		\item Strategies, payoffs, and equilibrium.
	\end{enumerate}
	\item Strengthandherculeanequilibrium. 
	\begin{enumerate}
		\item Definition 1 (Strength).
		\item Lemma 1.
		\item Definition 2 (Herculean Equilibrium).
	\end{enumerate}
	\item Auctions with two potential bidders.
	\begin{enumerate}
		\item Proposition 1.
		\item Proposition 1 establishes the existence of a herculean equi- librium, confirming the intuition that an equilibrium in which the strong bidder plays a lower entry cutoff should exist.
		\item Proposition 1 provides a sufficient condition on the CDFs’ shape for the game to have a unique equilibrium.
		\item Lemma 2.
		\item Lemma 2 further characterizes sufficient condition.
		\item Example 1 (Log-normal valuations).
		\begin{enumerate}
			\item Uniqueness under sufficiently high entry costs.
			\item Multiplicity and uniqueness under symmetry.
			\item Asymmetric auctions.
		\end{enumerate}
		\item Auctions with n potential bidders.
		\begin{enumerate}
			\item Lemma 3.
			\item Proposition 2.
			\item Asymmetric bidders.
			\item Ordered bidders. 
			\item Definition 3 (Ordered Auction).
			\item Lemma 4.
			\item Example 2.
			\item Proposition 3.
		\end{enumerate}
	\end{enumerate}

\end{enumerate}
\section{A model of market entry}
\begin{enumerate}
	\item The baseline model.
	\begin{enumerate}
		\item Set up.
		\item Main assumptions.
		\begin{enumerate}
			\item A 1 (Monotonicity).
			\item A 2 (Substitutes).
			\item A 3 (Costly and Interior Entry).
			\item A 4 (AffiliatedSignals).
		\end{enumerate}
		\item Example 3.
		\begin{enumerate}
			\item Linear model.
			\item SPA with partial information.
		\end{enumerate}
	\end{enumerate}
	\item Strategies, payoffs, and equilibrium.
	\begin{enumerate}
		\item Payoffs and strategies.
		\item Strength and herculean equilibrium.
		\item Definition 4 (Strength).
		\item Lemma 5.
		\item Definition 5 (Expected Profit Gain).
	\end{enumerate}
	\item Uniqueness with two groups of firms.
	\begin{enumerate}
		\item Proposition 4.
		\item Proposition 4 extends the existence of a herculean equilibrium result to the general framework in the two-group model.
		\item Proposition 4 also provides four conditions that need to be satisfied for equilibrium uniqueness—two conditions per group.
		\item Corollary 1.
		\item Example 4.
		\begin{enumerate}
			\item Linear model.
			\item SPA with partial information.
		\end{enumerate}
		\item Extensions.
		\begin{enumerate}
			\item A weaker sufficient condition.
			\item N groups of ordered entrants.
		\end{enumerate}
	\end{enumerate}
\end{enumerate}
\section{Concluding remarks}
\begin{enumerate}
	\item This article studies equilibrium uniqueness in static entry games with single-dimensional private information.
	\item This article focuses on entry games when firms’ entry decisions are strategic substitutes.
	\item The focus of this article is on static entry games with private information.
\end{enumerate}
\section*{Note}
I constructed the reverse outline manually. Papers in Economics in generak have very clear and strict structures. It begins with an introduction, followed by some illustrative examples to help understand the theoretical models if any. Then formally introduce the model. Present empirical results if any and end it by conclusions. 












\end{document}