\documentclass[12pt,one side]{article}\usepackage[utf8]{inputenc}\usepackage[a4 paper]{geometry}\usepackage{graphicx, setspace, appendix, mathrsfs, enumerate,amsmath, amsfonts, array, tabularx, longtable, rotating, caption, mathtools}\usepackage[english]{babel}\renewcommand{\baselinestretch}{1.25}\doublespacing\title{}\author{}\begin{document}\maketitle
\section{Question 1}
The switching costs from low to high are: $A < B < C$.\\
When $p_B$ changes from 15 to 10 in period 4, case C has less people switch from A to B compared to case B, therefore C has higher switching costs. In case A even when prices don't change there are switchers, while in case B and C there are no switchers when prices do not change. Therefore, case A has the lowest switching costs.

\section{Question 2}
\begin{enumerate}[1]
	\item
	The consumer will search for $k$ times
	\item
	\item
	Increasing $c$ by $\varepsilon$ reduces the probability of switching more, since the consumer needs to pay the search cost $c$ for every search and only need to pay the switch cost $s$ once if the consumer chose to switch.
	\item
\end{enumerate}

\end{document}