\documentclass[dvipsnames,mathserif]{beamer}
\setbeamertemplate{footline}[frame number]
\setbeamercolor{footline}{fg=black}
\setbeamerfont{footline}{series=\bfseries}
\usepackage{tikz}
\usepackage{xcolor}
\usepackage{graphicx, setspace, appendix, mathrsfs, amsmath, amsfonts,caption, mathtools}
\usepackage[english]{babel}
%\usetheme{Frankfurt}%1
\usetheme{Darmstadt}%1

% for RTL liste
\makeatletter
\newcommand{\RTListe}{\raggedleft\rightskip\leftm}
\newcommand{\leftm}{\@totalleftmargin}
\makeatother

% RTL frame title
\setbeamertemplate{frametitle}
{\vspace*{-1mm}
  \nointerlineskip
    \begin{beamercolorbox}[sep=0.3cm,ht=2.2em,wd=\paperwidth]{frametitle}
        \vbox{}\vskip-2ex%
        \strut\hskip1ex\insertframetitle\strut
        \vskip-0.8ex%
    \end{beamercolorbox}
}
% align subsection in toc
\makeatletter
\setbeamertemplate{subsection in toc}
{\leavevmode\rightskip=5ex%
  \llap{\raise0.1ex\beamer@usesphere{subsection number projected}{bigsphere}\kern1ex}%
  \inserttocsubsection\par%
}
\makeatother

% RTL triangle for itemize
\setbeamertemplate{itemize item}{\scriptsize\raise1.25pt\hbox{\donotcoloroutermaths$\blacktriangleleft$}} 

%\setbeamertemplate{itemize item}{\rule{4pt}{4pt}}

\defbeamertemplate{enumerate item}{square2}
{\LR{
    %
    \hbox{%
    \usebeamerfont*{item projected}%
    \usebeamercolor[bg]{item projected}%
    \vrule width2.25ex height1.85ex depth.4ex%
    \hskip-2.25ex%
    \hbox to2.25ex{%
      \hfil%
      {\color{fg}\insertenumlabel}%
      \hfil}%
  }%
}}

\setbeamertemplate{enumerate item}[square2]
\setbeamertemplate{navigation symbols}{}


\titlegraphic { 
\begin{tikzpicture}[overlay,remember picture, opacity=0.1,]
\node[] at (0, 2.9){
};\end{tikzpicture}}
\setbeamertemplate{caption}[numbered]
\begin{document}

\rightskip\rightmargin
\title{Beyond Truth-Telling: Preference Estimation with Centralized School Choice and College Admissions†}
\author{Gabrielle Fack, Julien Grenet, and Yinghua He}


%\institute{Boston College}
\footnotesize{\date{\today }


\begin{frame}
\maketitle
\end{frame}


%
\footnotesize \tableofcontents
%
\section{Introduction}
\begin{frame}{Introduction}
    \begin{itemize}
        \item Gale-Shapley deferred acceptance (DA) becomes the leading centralized machanism for the placement of students to public schools at every education level. One main reason is strategy-proofness.\\
        \item Is it reasonable to assume the submitted rank-ordered lists (ROLs) reveal students' true preferences over schools?\\
        \item The truth-telling assumption can be restrictive when students face only limited uncertainty. e.g. Schools rank students by test scores, students with a low test score might skip the highly selective schools in their ROLs.\\
        \item The analysis focus on the student-proposing Gale-Shapley deferred acceptance (DA)
        
    \end{itemize}
\end{frame}

\section{Contributions}
\begin{frame}{Contributions}
    \begin{itemize}
    \item The paper examines the implications of the truth-telling assumption in the situation with limited uncertainty.\\
    \item The paper proposes a set of novel empirical approaches that are theoretically founded, based on different identifying assumptions: weak truth-telling(WTT), stability and undominated strategies. It also provides statistical tests to guide the choice between these identifying assumptions.\\
    \item The paper evaluates the performance of each approach based on simulated and real-life data. It illustrate the main theoretical results with Monte Carlo simulations. And it applies the empirical approaches to school choice data from Paris.\\
    \end{itemize}  
\end{frame}

\section{Discussion}
\begin{frame}{Discussion}
    \begin{itemize}
        \item The theoretical and empirical results show that it can be rather restrictive to assume truth-telling. Instead, stability provides rich identifying information but weaker than WTT.\\
        \item Bonus point: the paper provides tests for different assumptions. Conditional on parametric assumption of the model.\\
        \item In table 7 it summarizes different features which make each assumption more plausible. Nice guidance for empirical work.\\
        \item Concrete theoretical foundations plus novel empirical approaches. Can extend to settings other than school choice.\\

    \end{itemize}
\end{frame}


\end{document}