\documentclass[12pt,one side]{article}\usepackage[utf8]{inputenc}\usepackage[a4 paper]{geometry}\usepackage{graphicx, setspace, appendix, mathrsfs, amsmath, amsfonts, array, enumerate,tabularx, longtable, rotating, caption, mathtools}\usepackage[english]{babel}\renewcommand{\baselinestretch}{1.25}\doublespacing\title{}\author{}\begin{document}\maketitle
\section{Question 1}
\begin{enumerate}[(a)]
	\item
	\textbf{Perfect competition}:\\
	\[(p-c)y(p) = \lambda, y(p_L) = 1, y(p_H) = \frac{1}{2}\]
	\[p_L = \lambda =1, p_H = 2\lambda = 2\]
	Plug it into low state demand and $Q_L = 100(10-p_L) = 900$. The residual demand for high state is
	\[RD(p_H,H) = 200(10-p_H)(1 - \frac{Q_L}{200(10-p_L)}) = 800\]
	Therefore, under perfect competition the equilibrium prices and quantities in low demand state and high demand state are (1,900) and (2,800), respectively.\\
	\textbf{Monopoly}:\\
	In low demand state,
	\[MR = 10 - \frac{Q}{50} = \lambda \rightarrow Q_L = 450, p_L = 5.5\]
	The residual demand for high state is
	\[RD(p_H,H) = 200(10-p_H)(1 - \frac{Q_L}{200(10-p_L)}) = 100(10-p_H)\]
	\[p_H = 10 - 0.01Q_H\]
	\[MR = 10 - 0.02Q_H = 2\lambda, Q_H = 400, p_H = 6\]
	Therefore, under monopoly the equilibrium prices and quantities in low demand state and high demand state are (5.5,450) and (6,400), respectively.\\
	\textbf{Discussion}: In this case price dispersion is higher under perfect competition than under monopoly, which coincides with Dana(1999). This is because when demand is linear, the pass through rate (PTR) of monopoly is $\frac{1}{2}$ while PTR of perfect competition is 1. In general as PTR increases, so does price dispersion.
	\item
	\textbf{Perfect competition}:\\
	\[p_L = \lambda = 1,p_H = 2\lambda = 2,Q_L = \frac{1600}{p_L^2} = 1600\]
	The residual demand of high state is:
	\[RD(p_H,H) = \frac{3200}{p_H^2}\times(1-\frac{Q_L}{\frac{3200}{p_L^2}}) = 400\]
	Therefore, under perfect competition the equilibrium prices and quantities in low demand state and high demand state are (1,1600) and (2,400), respectively.\\
	\textbf{Monopoly}:\\
	\[MR = \frac{1}{2}\sqrt{\frac{1600}{Q_L}} = \lambda, Q_L = 400, p_L = 2\]
	The residual demand in high state is:
	\[RD(p_H,H) = \frac{3200}{p_H^2}(1-\frac{Q_L}{\frac{3200}{p_L^2}}) = \frac{1600}{p_H^2}, p_H = \sqrt{\frac{1600}{Q_H}}\]
	\[MR = \frac{1}{2}\sqrt{\frac{1600}{Q_H}} = 2\lambda, Q_H = 100,p_H = 4\]
	Therefore, under monopoly the equilibrium prices and quantities in low demand state and high demand state are (2,400) and (4,100), respectively.\\
	\textbf{Discussion}: In this case, price dispersion is higher under monopoly than perfect competition. This is different from Dana(1999). The main reason is here the demand curve is not linear, it is convex. And the PTR under monopoly is 2, which is higher than the perfect competition PTR which is 1.  
	\item
	\textbf{Perfect competition}:\\
	Since $\lambda = 0$, $p_L = p_H = c$. $Q_L = 9$, $Q_H = 19$.
	\textbf{Monopoly}:\\
	\[MR = 10 - 2Q_L = c \rightarrow Q_L = 4.5, p_L = 5.5\]
	The residual demand in high state is:
	\[RD(p_H,H) = (20-p_H)(1-\frac{Q_L}{20-p_L}) = \frac{20}{29}(20-p_H)\]
	\[p_H = 20 - \frac{29}{20}Q_H, MR = 20 - \frac{29}{10}Q_H = c\]
	\[Q_H = \frac{190}{29}, p_H = 10.5\]
	Therefore, under monopoly the equilibrium prices and quantities in low demand state and high demand state are (5.5,4.5) and $(10.5,\frac{190}{29})$, respectively.\\
	\textbf{Discussion}: In Dana(1999) when $\lambda = 0$ there is no price dispersion under both perfect competition and monopoly. This case is different from Dana(1999) since there is price dispersion under monopoly. The main reason is that Dana(1999) makes the functional form assumption that $D(p\theta) = \theta D(p)$. The demand fuction in (c) clealy does not satisfy this assumption. Therefore, even when capacity cost is zero demand uncertainty alone can still cause price dispersion when the demand fuction is non-seperable.
\end{enumerate}

\section{Question 2}
\begin{enumerate}[(a)]
	\item
	According to Tirole(1988), the upper bound of the welfare change from third-degree price discrimination to uniform pricing is
	\[\Delta W = (\overline{p} - c)(\Sigma_i\Delta q_i)\]
	where $\overline{p}$ is the monopoly uniform price. $\Delta q_i$ is the change in demand. From the abovr inequality $\Delta W$ is positive only when the total quantity change is positive. Intuitively, price discrimination causes different customers to have different marginal rates of substitution. Given a fixed output, some goods are transfered from high value consumers to low value consumers with a deadweight loss. The total output needs to increase to have welfare gains under price discrimination.
	\item
	When the utility function is quasi-linear, the social welfare function puts equal weights on everyone regardless of their income.
	

\end{enumerate}
\end{document}