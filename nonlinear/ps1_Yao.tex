\documentclass[12pt,one side]{article}\usepackage[utf8]{inputenc}\usepackage[a4 paper]{geometry}\usepackage{graphicx, setspace, appendix, mathrsfs, amsmath, amsfonts, array, tabularx, longtable, rotating, caption, mathtools}\usepackage[english]{babel}\renewcommand{\baselinestretch}{1.25}\doublespacing\title{IO2 problem set 1}\author{Yao Luo}\begin{document}\maketitle
\section{Question 1}
The buyer's utility maximization problem is:
\[max_{q(\theta), T(\theta)}\int_{\underline{\theta}}^{\overline{\theta}}[V(q(\theta)) - T(\theta)]f(\theta)d\theta\]
subject to\\
(1) Individual Rationality constraint (IR):
\[T(\theta) - C(q(\theta),\theta) \geq 0 \; \forall \theta\]
(2) Incentive Compatibility constraint (IC):
\[T(\theta) - C(q(\theta),\theta) \geq T(\theta^{'}) - C(q(\theta^{'}),\theta)\;\forall \theta,\theta^{'}\]
Denote $\pi(\theta,\hat{\theta}) = T(\hat{\theta}) - C(q(\hat{\theta}),\theta)$
To satisfy the local IC, the FOC with respect to $\hat{\theta}$ is zero.
\[\frac{d\pi(\theta,\hat{\theta})}{d\theta}= -C_{\theta}(q,\theta) < 0\]
\[\pi(\theta,\hat{\theta}) \equiv \pi(\theta) = \pi(\overline{\theta}) + \int_{\theta}^{\overline{\theta}} C_{\theta}(q,\theta)\]
Since the IR constraint binds for the highest type if IC holds, $\pi(\overline{\theta}) = 0$. 
\[T(\theta) = C(q,\theta) + \int_{\theta}^{\overline{\theta}} C_{\theta}(q,\theta)\]
Now the relaxed problem of the buyer is:
\[\underset{q(\theta)}{max}\int_{\underline{\theta}}^{\overline{\theta}}[V(q(\theta)) - C(q,\theta) - \int_{\theta}^{\overline{\theta}} C_{\theta}(q,\theta)]f(\theta)d\theta\]
Using integration by parts
	\begin{align*}
	\int_{\underline{\theta}}^{\overline{\theta}}\int_{\theta}^{\overline{\theta}} C_{\theta}(q(x),x)f(\theta)dxd\theta &= F(\theta)\int_{\theta}^{\overline{\theta}}C_{\theta}(q(x),x)dx\Bigg|_{\underline{\theta}}^{\overline{\theta}} + \int_{\underline{\theta}}^{\overline{\theta}}C_{\theta}(q(\theta),\theta)F(\theta)d\theta\\
    &= \int_{\underline{\theta}}^{\overline{\theta}} F(\theta)C_{\theta}(q(\theta),\theta)d\theta
	\end{align*}
	The relaxed problem is simplified to be:
	\[\underset{q(\theta)}{max}\int_{\underline{\theta}}^{\overline{\theta}}[V(q(\theta)) - C(q(\theta),\theta) - \frac{F(\theta)}{f(\theta)}C_{\theta}(q(\theta),\theta)]f(\theta)d\theta\]
	Denote \[\psi(q,\theta) = V(q(\theta)) - C(q(\theta),\theta) - \frac{F(\theta)}{f(\theta)}C_{\theta}(q(\theta),\theta) \]
	To solve this unconstrained problem, the following conditions should hold:\\
	(1) FOC
	\[V_{q}(q) - C_{q}(q,\theta) -\frac{F(\theta)}{f(\theta)}C_{\theta q}(q,\theta) = 0\]
	(2) SOC
	\[V_{qq} - C_{qq} - \frac{F(\theta)}{f(\theta)}C_{\theta qq}(q,\theta) \leq 0\]
	which is satisfied if $C_{\theta qq} \geq 0$\\
	(3) Global IC\\
	A sufficient condition for global IC is $\frac{\partial^{2}\pi(\theta, \hat{\theta})}{\partial \hat{\theta}\partial\theta} \geq 0$. Because $\frac{\partial\pi(\theta,\theta)}{\partial\hat{\theta}} = 0$, if $\theta > \hat{\theta}$, $\frac{\partial\pi(\theta,\hat{\theta})}{\partial\hat{\theta}} \geq 0$ so sellers who reported their type below their true type want to increase their reported type $\hat{\theta}$. And if $\theta < \hat{\theta}$, $\frac{\partial\pi(\theta,\hat{\theta})}{\partial\hat{\theta}} \leq 0$ so sellers who reported their type above their true type want to decrease their reported type $\hat{\theta}$. Since
	\[\frac{\partial^{2}\pi(\theta, \hat{\theta})}{\partial \hat{\theta}\partial\theta} = - C_{q\theta}(q(\hat{\theta}),\theta)\frac{dq(\hat{\theta})}{d\hat{\theta}}\]
	plus $C_{q\theta} > 0$, the sufficient condition is $\frac{dq(\hat{\theta})}{d\hat{\theta}} \leq 0$. By monotone comparative statics, $\frac{\partial^{2}\psi(q,\theta)}{\partial q\partial\theta} \leq 0$.
	\[-C_{q\theta}(q,\theta) - \frac{d}{d\theta}[\frac{F(\theta)}{f(\theta)}]C_{q\theta}(q,\theta) - \frac{F(\theta)}{f(\theta)}C_{q\theta\theta} \leq 0\]
	To satisfy the above inequality, we need to assume $C_{q\theta\theta} \geq 0$.\\
	From the seller's maximization problem $q = argmax(P(q) - C(q,\theta))$, we have $P'(q) = C_{q}(q,\theta)$. From the buyer's FOC we get: 
	\[P'(q) = C_{q}(q,\theta) = V_{q}(q) - \frac{F(\theta)}{f(\theta)}C_{q\theta}(q,\theta)\]
	Since $C_{q\theta}(q,\theta)$ is positive, the monopsony purchasing price is lower than the first best case where marginal price equals marginal cost. Because $C_{qq}(q) \geq 0$, the monopsony quantity $q(\theta)$ is lower than the first best quantity $q^{FB}$. Since $F(\underline{\theta}) = 0$, the lowest type is not distorted so $q(\underline{\theta}) = q^{FB}$.


\section{Question 2}
\begin{enumerate}
	\item
	The rental company's maximization problem is:
	\[\underset{q^B,p^B,q^T,p^T}{max}\frac{2}{3}[p^B - C(q^B)] + \frac{1}{3}[p^T-C(q^T)]\]
	subject to:
	\[V^T(q^T) - p^T \geq 0 \]
	\[V^B(q^B) - p^B \geq 0\]
	\[V^B(q^B) - p^B \geq V^B(q^T) - p^T\]
	\[V^T(q^T) - p^T \geq V^T(q^B) - p^B\]
	Since $V^B(q) - V^T(q) \geq 0 \; \forall q$, the IR condition for the T type is binding:
	\[V^T(q^T) = p^T\]
	The IC constraint for B type is also binding:
	\[V^B(q^B) - p^B = V^B(q^T) - p^T\]
	The relax problem of ignoring the IC constraint for the T type is:
	\[\underset{q^B,q^T}{max}\frac{2}{3}[V^B(q^B) - V^B(q^T) +V^T(q^T)- C(q^B)] + \frac{1}{3}[V^T(q^T) - C(q^T)]\]
	And we need to check the IC for type T holds:
	\[V^B(q^B) - V^B(q^T) \geq V^T(q^B) - V^T(q^T)\]
	Denote 
	\[\pi_B = V^B(q^B) - C(q^B)\]
	\[\pi_T = V^T(q^T) - \frac{2}{3}V^B(q^T) - \frac{1}{3}C(q^T)\]
	\[\frac{\partial\pi_B}{\partial q^B} = 30 - 3q^B - 0.05 = 0 \rightarrow q^B = \frac{599}{60}\]
	Compare it with $q^B = 10$, we get that $\pi_B$ is maximized when $q^B = \frac{599}{60} \approx 9.98$. 

	(1)$q^T \geq 500$
	\[\frac{\partial\pi_T}{\partial q^T} = -\frac{1}{3}\times0.05\]
	(2)$10 \leq q^T \leq 500$
	\[\frac{\partial\pi_T}{\partial q^T} = \frac{1}{2} - \frac{1}{1000}q^T - \frac{1}{3}0.05 = 0 \rightarrow q^T = \frac{2900}{6} \approx 483.33\]
	(3) $0 \leq q^T \leq 10$
	\[\frac{\partial\pi_T}{\partial q^T} = \frac{1}{2} - \frac{1}{1000}q^T - \frac{2}{3}[30 - 3q^T] -\frac{1}{3}0.05 = 0 \rightarrow q^T = \frac{117100}{11994} \approx 9.76 \]
	Therefore, $\pi_T$ is maximized when $q^T = 483.33$.
	The optimal price schedules are:
	\[q^B \approx 9.98, q^T \approx 483.33, p^B = \frac{898997}{7200} \approx 124.86, p^T = \frac{4495}{36} \approx 124.86\]
	Type T does not have the incentive to be type B consider the two prices are basically the same. So IC for type T is satisfied.

	\item Question a does not coincide witht the analysis in class. Mainly because the single crossing assumption $V_{\theta q} \geq 0$ is violated here. 
	\item Now we don't have the nice property of $V^B(q) - V^T(q) \geq 0 \; \forall q$, IR for T type binds does not imply IR for B type binds. The firm's profit is maximized when both IR constraints are binding. Therefore, 
	\[\frac{\partial\pi}{\partial q^B} = \frac{2}{3}[30 - 3q^B - 0.05] = 0 \rightarrow q^B = \frac{599}{60}\]
	\[\frac{\partial\pi}{\partial q^T} = \frac{1}{3}[\frac{1}{2} - \frac{1}{1000}q^T - 0.05] = 0 \rightarrow q^T = 450\]
	$p^B \approx 150$, $p^T = 123.75$. Check the IC constraints:
	IC for type B:
	\[30\times450 - 1.5\times 450^2 - 123.75 \approx -2.9\times10^5 < 0\]
	IC for type T:
	\[\frac{1}{2}\times\frac{599}{60} - \frac{1}{2}\frac{1}{1000}\times(\frac{599}{60})^2 - 150 \approx -145 <0\]
	So both IC constraints are satisfied.
\end{enumerate}
\section{Question 3}
The single crossing assumption might be reasonable for cable TV markets in the sense that higher type (richer) consumers are more willing to pay for cable and have a higher consumption than low type. However, in the rural food staples markets, low type (poor family) tend to have more children and may consume more food than high type (rich family). In this case the single crossing assumption does not make sense. Moreover, in the rental car market above, the preferences do not satisfy single crossing. Therefore, whether single crossing is a reasonable assumption to make depends on the exact market context.

\end{document}