\documentclass[12pt,oneside]{article}\usepackage[utf8]{inputenc}\usepackage[a4paper]{geometry}\usepackage{graphicx,setspace,appendix,mathrsfs,amsmath,amsfonts,array,tabularx,longtable,rotating,caption,mathtools,natbib}\usepackage[english]{babel}\renewcommand{\baselinestretch}{1.25}\bibliographystyle{agsm}\doublespacing\usepackage{hyperref}\title{ Endogenous information acquisition in matching markets: China's college admission mechanisms}\author{Yao Luo}\begin{document}\maketitle


\citet{grenet_preference_2022} question the common assumption in the matching literature that agents have full information on their own preferences and conclude that students' costly discovery of preferences exist. Similarly, \cite{bade_serial_2015} don't believe that agents precisely know their preferences and introduce the assumption of endogenous information acquisition into otherwise standard house allocation problems. Similar work includes \cite{corcoran_leveling_2018}, \cite{dynarski_closing_2018} and \cite{grenet_decentralizing_2022}. Therefore, market designers should pay attention to the acquisition and flow of information besides stability.

\cite{immorlica_information_2020} defines regret-free stable outcomes in the context of college admission and school choice, and present an impossibility result that regret-free stable matching outcomes cannot be achieved because of indformation deadlocks. However, approximately regret-free stable outcomes could be achieved by providing external historical data with respect to perturbed capacities. Follow the work by \cite{azevedo_2016}, \cite{immorlica_information_2020} also shows that finding regret-free stable outcomes is equivalent to finding market-clearing cutoffs. However, China's two college admission mechanisms don't pertub capacities. So we know certainly some students have regret because of the costly information acquisition. 

Most provinces in China use the Parallel admission mechanism (PA). It is a variant of direct serial dictatorship (DirSD) with the restriction on the length of the submitted rank-ordered list (ROL). The important feature is that students submit their ROL simultaneously. Inner Mongolia uses a dynamic admission mechanism (IM), \cite{bai_analysis_2022} has a detailed description. It can be seen as sequential serial dictatorship (SeqSD) with time constraints and the restriction that groups of students move sequentially instead of individuals.

Both PA and IM provides historical cutoff scores, but IM offers additional information regarding matching outcomes. Normally we think more information helps, so my first guess is that IM achieves higher student welfare than PA. However, in 2025 Inner Mongolia will give up IM and use PA. This seems counterintuitive to me, especially since the government has invested a lot of money on IM mechenism and has used it since 2007. 

This puzzling fact leads to my research question: Does PA indeed achieves higher student welfare than IM? If so why does the additional information provided by IM harm student welfare instead? Who are the winners and losers in this policy change? Maybe the time constraints in IM (one hour per group) make the price discovery process too costly? Maybe the information communication in IM is not effective? Maybe the additional information is too noisy ? Or maybe IM is indeed better than PA and the policy change is purely of political intention. Moreover, are there better ways to communicate information to students to improve matching outcomes of IM?

\section{Literature Review}

Theoretically, \cite{hakimov_costly_2023} examines the cost of information acquisition in centralized matching markets and how to reduce wasteful information acquisition. Similarly, \cite{hakimov_costly_2023} look at information acquisition in matching markets but with a focus on the role of price discovery. 

\cite{chen_chinese_2017} and \cite{chen_chinese_2019} compares both theoretically and experimentally the matching outcomes of PA and DA under the assumption of complete information and find that DA is better than PA in terms of stability.

\cite{gong_2023} shows experimentally that IM mechanism achieves similar stability as DA mechanism and similar efficiency as Boston mechanism when the correlation of preferences is low. They assume incomplete information.

\cite{chen_time-constrained_2022} focus on the  impact of time-constraint imposed by the authority, and find the matching outcomes of time-constrained dynamic mechanism (TCDM) can be worse than the stduent-proposing deferred acceptance (DA) mechanism.  

\cite{bo_iterative_2022}look at the university admissions in Brazil and show that dynamic mechanism can outperform the standard ones since matchings are implemented sequentially and students can revise their choices after receiving information or feedback about their current allocations. \cite{hakimov_costly_2023} shows both theoretically and experimentally that a sequential serial dictatorship mechanism leads to higher student welfare than a direct serial dictatorship mechanism. 

\section{Contribution}
This paper will provide empirical and experimental comparison of real-life implementation of DirSD (PA) and SeqSD (IM).

This paper sheds light on the importance of information flows in market design and validate costly endogenous information acquisition.

\section{Data}
Students' exam scores, rank, ethnicity, gender and address. University's past cutoff scores and address.

\section{Empirical Strategy}

        \[y_i = \alpha_0 + \alpha_1 X_i + \beta Y_{2025} + \varepsilon_i\]
where $y_i$ is college prestige index calculated by dividing college's rank (determined by the average cutoff scores of year 2022 and 2023) by the total number of colleges. $X_i$ include students' gender, ethnicity, rank by exam scores being normalized to be within (0,1) and an indicator $H_i$ which equals to one if the student was admitted by a college in his or her home province and zero otherwise. $Y_{2025}$ is a year dummy and can be seen as exogenous if we restrict our sample to only first-time exam takers. Since in China the time students take the Gaokao is purely determined by their birth date, which is independent of the policy change.

I implicitly assume that students only care about the prestige of the college. For robustness checks, I calculate the prestige index using the QS ranking. 



\section{Experiment design}
\subsection{General setup}
Exam scores are randomly and independently drawn from the uniform distribution between 0 and 100. 30 students competing for 15 seats in 10 colleges. Students are told that any ranking of the university is equally possible and preferences are randomly drawn from the space of rankings. Students only know their own exam scores and ranks. University's quotas and historical cutoffs are common knowledge. Historical cutoffs are gotten by running DirSD. Students need to pay search costs to know their own preferences, which are private knowledge. Students receive more rewards for being assigned to more preferred university.
\subsection{Environments}
Dimension 1: The degree of correlation of preferences among students. One tier, two tiers and three tiers. It is common knowledge that tier one colleges are strictly better than tier two ones, and tier two is better than tier three. \\
Dimension 2: The cost of information acquisiton. Low and high costs.
\subsection{Predictions}
Hypothesis 1: Lower-ranked students gain more from IM compared to PA.\\
 Hypothesis 2: IM has a higher probability of being unmatched.\\
 Hypothesis 3: Students are more likely to oversearch in IM.\\
Hypothesis 4: Welfare: on average IM $<$ PA, but not hold for all students.\\
 Hypothesis 5: Smaller group size produces better matching outcomes.\\
 Hypothesis 6: Increasing the ROL in PA improves the matching outcomes.


\section{Caveats}
Over the years people have became more rational and care more about majors. The dispersion of admitted students' exam scores within each college is increasing. Moreover, the universities determine the cutoff scores by their total quota instead of quota of each major. In many cases some major might have too many applicants while some have too few. The university would reassign the lower-ranked students to the unfilled seats of another major even though the stduents do not want it. This sometimes result in terrible assignment results and many students choose to go back to high school and retake the Gaokao. Matching students to their prefered majors might be a better objective to achieve. But the current literature has not looked in this direction.

One method I propose to accommodate students' major preferences is to allow for full rank of all the majors the university offers instead of only rank 5 majors in each college choice. The advantage is that students decrease the risk of being assigned to a major they hate. However, it also increases students' cost of information acquisition. The overall impact on students' welfare is ambiguous.

\newpage
\bibliography{college}
\end{document}